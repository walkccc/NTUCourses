% !TEX program = xelatex
\documentclass{article}
\usepackage{/Users/jay/LaTeX/cs}
\usepackage{/Users/jay/LaTeX/codelist}

\newcommand{\hmwkClass}{Computer Architecture, Fall 2017}
\newcommand{\hmwkTitle}{Project 1 report}
\newcommand{\hmwkDueDate}{DECEMBER 11, 2017}
\newcommand{\tb}{\textbf}

\begin{document}

\thispagestyle{empty}
\section*{\hmwkClass \\
    \normalsize{\hmwkTitle} \\
    \normalsize{DUE DATE: \hmwkDueDate}
}

\section{Members \& Team Work} 
\begin{itemize}
    \item b03902125 郭曉嵐
    \begin{itemize}
        \item Debugging.
        \item Fixing some data hazard error.
    \end{itemize}

    \item b03902127 林映廷
    \begin{itemize}
        \item Linking the module with \textit{CPU.v}.
        \item Handling with \textit{flush} and \textit{stall} in \textit{HazardDetection.v}.
    \end{itemize}

    \item b03902129 陳鵬宇
    \begin{itemize}
        \item Linking the module with \textit{CPU.v}.
        \item The implementation of \textit{Forwading.v}.
        \item The implementation of each sub module.
    \end{itemize}

\end{itemize}

\section{Implementation of the Pipeline CPU}

We use \textit{testbench.v} to see the output; \textit{CPU.v} to build the whole datapath.\\

The \textit{CPU.v} is the core of the whole program, which takes two inputs \textit{clk\_i} since we change the value when \textit{positive clock edge} and \textit{start\_i} since we need to know when the program is starting, so we set the \textit{start\_i} to 1 when the clock time begins for one-fourth of the \textit{CYCLE\_TIME} to ensure that the start signal is valid at the right time. 

\begin{itemize}
    \item \textbf{testbench}: Do the following by specs:
    \begin{itemize}
        \item Initialize storage units.
        \item Load "instruction.txt" into instruction memory.
        \item Create clock signal.
        \item Output cycle count in each cycle.
        \item Output Register File \& Data Memory in each cycle.
        \item Print result to "output.txt".
        \item Handle when to \textit{stall} or \textit{flush}.
    \end{itemize}

    \item \textbf{CPU}: Declare the $wire$ and $reg$ type variables in $CPU.v$ since there may be an output of a module connecting to several inputs of other modules. If we don't declare these variables globally, the codes will look tedious and hard to read. 
    And classify the registers as four types:
    \begin{itemize}
        \item IF/ID registers
        \item ID/EX registers
        \item EX/MEM registers
        \item MEM/WB registers
    \end{itemize}    
\end{itemize}

\newpage
\section{Implementation and Explanation of Each Module}
    \subsection*{IF stage}
    \begin{itemize}
        \item \textbf{Instruction\_Memory}: Save the instructions.
        \item \textbf{PC}: \textit{PC} module:
        \begin{itemize}
            \item if \textit{hazard} doesn't occur and \textit{start\_i} is set, updates the PC value.
            \item else if \textit{start\_i} is not set, clears the \textit{PC value} to zeros.
        \end{itemize}
        \item \textbf{Add\_PC}: Add the \textit{instAddr} for 4.
        \item \textbf{MUX\_Branch}: Determine whether \textit{branch} to new address or not.
        \item \textbf{MUX\_Jump}: Determine whether \textit{jump} to new address or not. 
    \end{itemize}

    \subsection*{ID stage}
    \begin{itemize}
        \item \textbf{Registers}: Save the data of registers.
        \item \textbf{ADD}: Add the \textit{shifted address} and \textit{IFID\_pc} together.
        \item \textbf{ShiftJump}: Shift the \textit{jump address} left by two bits, and concatenate with two zeros.
        \item \textbf{ShiftLeft2}: Shift the \textit{address} left by two bits.
        \item \textbf{Sign\_Extend}: Extend the 16-bits \textit{address} to 32 bits by filling in the $[31:16]$ bits as the first bit of the input address, i.e., \textit{address$[15]$}.
        \item \textbf{Equal}: Determine whether \textit{RS\_data} and \textit{RT\_data} are equal or not.
        \item \textbf{Control}: Set the following control units to 0 or 1 by the 6-bits \textit{op\_i$[5:0]$}. 
        \begin{itemize}
            \item $RegWrite$
            \item $MemtoReg$
            \item $Branch$
            \item $MemRead$
            \item $MemWrite$
            \item $RegDst$
            \item $ALUOp$
            \item $ALUSrc$
            \item $Jump$
        \end{itemize}
        \item \textbf{MUX\_CtrlUnit}: Determine whether to \textit{stall} the pipeline or not.        
        \item \textbf{MUX\_RS}: Determine whether forwarding the \textit{RSdata} by the \textit{Forwarding} module.
        \item \textbf{MUX\_RT}: Determine whether forwarding the \textit{RTdata} by the \textit{Forwarding} module.
        \item \textbf{HazardDetection}: Detect the hazard. 
        \begin{itemize}
            \item if \textit{MemRead\_i} is set, it means that \textit{lw} may incurs some hazard, thus we have to \textit{stall} the pipeline for one clock cycle but do not \textit{flush} the pipeline. 
                  Also we set the \textit{pc\_hazard\_o} to 0.  
            \item else if the control units \textit{Jump\_i} or \textit{Branch\_i} \& \textit{Equal\_i} are set, we not only \textit{stall} the pipeline but \textit{flush} it.
                  Also we set the \textit{pc\_hazard\_o} to 1.
            \item else we continue our stages as before and set \textit{stall\_o, flush\_o} and \textit{pc\_hazard\_o} to 0.
        \end{itemize}
    \end{itemize}

    \newpage
    \subsection*{EX stage}
    \begin{itemize}
        \item \textbf{ALU}: Do the ALU operation with two data, and also assign the \textit{zero} value to 0 or 1.
        \item \textbf{ALU\_Controlt}: Determine the \textit{ALU operation} by the 6-bits \textit{function} field.
        \item \textbf{MUX\_RegisterRd}: Decide the \textit{RegisterRd} between \textit{inst$[20:16]$} and \textit{inst$[15:11]$}.
        \item \textbf{MUX\_ForwardA}: Determine the input of \textit{ALU.data1\_i} by the \textit{Forwarding} module. 
        \item \textbf{MUX\_ForwardB}: Determine the input of \textit{tmpALUdata2} by the \textit{Forwarding} module.
        \item \textbf{MUX\_ALUSrc2}: Determine whether the input of \textit{ALU.data2\_i} is \textit{tmpALUdata2} or the \textit{sign-extened data}.
        \item \textbf{Forwarding}: Determine the value of \textit{ForwardA} and \textit{ForwardB} by the following two data hazard:
        \begin{enumerate}
            \item \textit{EX hazard}:
            \begin{itemize}    
                \item if (EX/MEM.RegWrite 

                      and (EX/MEM.RegisterRd $\ne$ 0) 

                      and (EX/MEM.RegisterRd = ID/EX.RegisterRs)) ForwardA = 10
                \item if (EX/MEM.RegWrite 
                    
                      and (EX/MEM.RegisterRd $\ne$ 0) 
                      
                      and (EX/MEM.RegisterRd = ID/EX.RegisterRt)) ForwardB = 10
            \end{itemize}
            \item \textit{MEM hazard}:
            \begin{itemize}
                \item if (MEM/WB.RegWrite 
                
                      and (MEM/WB.RegisterRd $\ne$ 0) 
                    
                      and (MEM/WB.RegisterRd = ID/EX.RegisterRs)) ForwardA = 01
                
                \item if (MEM/WB.RegWrite 
                    
                      and (MEM/WB.RegisterRd $\ne$ 0) 
                      
                      and (MEM/WB.RegisterRd = ID/EX.RegisterRt)) ForwardB = 01
            \end{itemize}
        \end{enumerate}
    \end{itemize}

    \subsection*{MEM stage}
    \begin{itemize}
        \item \textbf{Data\_Memory}: Save the data of the memory.
    \end{itemize}

    \subsection*{WB stage}
    \begin{itemize}
        \item \textbf{MUX\_RegWriteData}: Determine the data written to the register is \textit{MEMWB\_ALUResult} or \textit{MEMWB\_ReadData}.
    \end{itemize}
    
    \newpage
    \section{Problems and Solution of the Project}
    To implement a such huge five stages pipeline is not easy, 
    we have to carefully connect each input and output in a correct manner. 
    Because we are not that familiar with \textit{verilog}, 
    we \textit{Google} a lot to \textit{fetch} the critical information we need. \\

    On the one hand, we took a long time to figure out the difference between "<=" and "=". 
    We use "<=" in a common situation and sequential logic and "=" in a combinational logic. 
    On the other hand, we also took some time to know the difference between \textit{reg} and \textit{wire}. 
    Following are the explanations of them we Googled: 

    \begin{itemize}
        \item \textbf{Wire} \\
        Wires are used for connecting different elements. 
        They can be treated as physical wires. 
        They can be read or \textit{assigned}. 
        No values get stored in them. They need to be driven by either continuous assign statement or from a port of a module. 
        \item \textbf{Reg} \\
        Contrary to their name, regs don't necessarily correspond to 
         physical registers. They represent data storage elements in
         Verilog/SystemVerilog. They retain their value till next value is
         assigned to them (not through assign statement). 
    \end{itemize}
    
    It is also very difficult to debug because there are \textit{twenty} .v files in the folder. 
    Our eyes really suffered a lot. 
    We inserted the following codes to generate the .vcd file for conveniently debugging: \\

    \hspace{10mm}\$dumpfile("project.vcd"); \vskip0mm
    \hspace{10mm}\$dumpvars \\

    By tracing the logic of the .vcd file, it can significantly increase the speed of debugging.

\end{document}